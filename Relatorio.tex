% Created 2019-10-29 ter 23:24
% Intended LaTeX compiler: pdflatex
\documentclass[11pt]{article}
\usepackage[utf8]{inputenc}
\usepackage[T1]{fontenc}
\usepackage{graphicx}
\usepackage{grffile}
\usepackage{longtable}
\usepackage{wrapfig}
\usepackage{rotating}
\usepackage[normalem]{ulem}
\usepackage{amsmath}
\usepackage{textcomp}
\usepackage{amssymb}
\usepackage{capt-of}
\usepackage{hyperref}
\usepackage[brazilian, ]{babel}
\author{Aline Freire de Rezende, Gilberto Lopes Inácio Filho}
\date{\today}
\title{Integração Numérica com Quadratura Adaptativa}
\hypersetup{
 pdfauthor={Aline Freire de Rezende, Gilberto Lopes Inácio Filho},
 pdftitle={Integração Numérica com Quadratura Adaptativa},
 pdfkeywords={},
 pdfsubject={},
 pdfcreator={Emacs 26.3 (Org mode 9.2.6)}, 
 pdflang={Pt_Br}}
\begin{document}

\maketitle
\tableofcontents


\section{Introdução}
\label{sec:orga1a292c}
O trabalho consiste na implementação de um algoritmo concorrente que
calcule a integral numérica de uma função usando a Regra do Ponto Médio,
com a estratégia de quadratura adpatativa.

A regra se baseia no cálculo da área do retângulo delimitado pelo
intervalo \([a, b]\), com altura \(f(x)\) calculada no ponto médio do intervalo.
Ao se dividir o problema em um numero suficientemente grande de subproblemas,
obtém-se uma boa aproximação da integral.

A estratégia de quadratura adaptativa é usada quando, ao invés de se fixar
um número de divisões, os seguintes passos são efetuados:

\begin{enumerate}
\item Calcula-se a área original, com base no intervalo \([a, b]\) e altura dada
por \(f(m)\), sendo \(m\) o ponto médio.
\item Divide-se o intervalo original em dois, e calcula-se a área nestes dois
subintervalos \([a, m]\) e \([m, b]\), com altura em seus respectivos pontos médios.
\item Compara-se a área original com a soma das áreas dos subproblemas. Se a
diferença entre elas for suficientemente pequena, considera-se a área maior
uma boa aproximação. Caso contrário, o processo é repetido recursivamente
nos subintervalos.
\end{enumerate}

\section{Estrutura do código}
\label{sec:org32c5699}
O código consiste nas aplicações de teste:
\begin{itemize}
\item Uma para cada função, executando os algoritmos concorrentes e o sequencial, e imprimindo os resultados calculados e os tempos obtidos.
\item Uma aplicação geral, que executa os algoritmos em todas as funções, imprime os resultados, os tempos gastos e o ganho dos algoritmos concorrentes.
\end{itemize}

Além disso o projeto possui arquivos cabeçalho referentes a:
\begin{itemize}
\item Cada um dos algoritmos de solução
\item A estrutura de pilha.
\item Um utilitário para o cálculo de áreas.
\item Um utilitário para a medição do tempo.
\item Um mapeamento das funções matemáticas de entrada em código C.
\end{itemize}

\section{Algoritmo Sequencial}
\label{sec:org8709751}
O algoritmo implementado na versão sequencial consiste num método recursivo simples de quadratura.
A função recebe como entrada os limites \([l, r]\) do intervalo visado,
o erro máximo de referência, e a função \(f(x)\) que se deseja integrar.
Na chamada inicial, o intervalo passado pra função é o intervalo inicial \([a, b]\).

\begin{itemize}
\item A área do \textbf{retângulo maior} \(q\) é calculada tendo como base a distância do intervalo \([l, r]\),
e a altura sendo definida como \(f(m)\) onde \(m\) é o ponto médio do intervalo.
\item A área do \textbf{retângulo esquerdo} \(lq\) tem como referência a base definida pelo subintervalo \([l, m]\),
e a altura \(f(m_{e})\) onde \(m_{e}\) é o ponto médio do subintervalo.
\item De modo análogo, a área do \textbf{retângulo direito} \(rq\) é obtida considerando-se o subintervalo \([m, r]\)
e a altura \(f(m_{d})\), com \(m_{d}\) sendo o ponto médio deste subintervalo.
\item Calculadas as áreas, encontra-se a diferença \(e\) entre a área do retângulo maior e a soma
das áreas dos retângulos menores, \(q - (lq + rq)\).
\end{itemize}

Se a diferença \(e\) for menor do que o erro máximo definido, considera-se a área do retângulo
maior uma aproximação boa o suficiente para este intervalo, e a função retorna o valor \(q\).
Caso contrário, a função retorna a soma das suas chamadas recursivas, uma no subintervalo
\([l, m]\) e a outra no subintervalo \([m, r]\).

Ao final de todas as chamadas recursivas, a chamada original retornará a soma de todas as
subaproximações consideradas satisfatórias, que representa por sua vez a aproximação da integral
definida da função \(f(x)\) no intervalo \([a, b]\).

\section{Algoritmo Concorrente}
\label{sec:org27d69a8}
Na implementação da solução concorrente, várias estratégias foram consideradas com o objetivo
de garantir o balanceamento de carga e o máximo desempenho possível do código. Apesar disso,
não houve êxito em implementar de maneira satisfatória e correta uma solução que contemplasse
os dois objetivos.

Por isso, foi decidido ilustrar as duas possibilidades que obtiveram o maior grau
de sucesso, com outras considerações sobre alternativas sendo feitas ao final do relatório.

Para os algoritmos concorrentes num geral, foi definido o conceito de uma \emph{tarefa}, que representa
um intervalo \([l, r]\) cuja área com altura \(f(m)\) no ponto médio será avaliada.

\subsection{Pilha}
\label{sec:orgd9bc670}
Para a execução desta solução, foi necessário o uso de uma pilha de tarefas. Além disso,
para melhorar o desempenho o conceito de tarefa foi estendido, de modo que a área maior
do intervalo, previamente calculada, é armazenada na tarefa para ser usada na próxima
execução.

O código foi implementado usando uma função envelope, que inicializa as variáveis globais de
uso comum pelas threads com os valores iniciais apropriados, empilha a tarefa inicial
representada pelo intervalo \([a, b]\) e sua \textbf{área maior}, dispara as threads, e após os seus
términos soma seus resultados parciais ao resultado total.

A lógica de execução das threads se baseia num laço de repetição indefinido. Sua parada é
determinada quando não houverem mais tarefas na pilha e todas as threads estiverem inativas.
O fluxo pode ser entendido da seguinte maneira:

\begin{itemize}
\item A thread avalia se é necessário retirar uma tarefa da pilha. Caso ela tenha divido uma tarefa
anteriormente, uma subtarefa terá sido inserida na pilha e outra terá sido retida pela thread,
de modo que uma retirada não é necessária.
\begin{itemize}
\item Caso seja necessário, a thread verifica se a pilha está vazia. Se estiver e todas as threads
estiverem inativas, a thread sai do laço principal de repetição e armazena seu resultado
parcial num vetor global. Se ainda houver threads ativas, a thread se bloqueia até que haja
uma tarefa na pilha. Quando ela voltar à execução, retirará uma tarefa.
\end{itemize}
\item A thread calcula as áreas dos subintervalos referentes às metades do intervalo original
e compara a soma destas com a área maior anteriormente calculada. 
\begin{itemize}
\item Se a diferença for maior do que o erro máximo definido, a thread divide o intervalo da tarefa
em duas metades e cria duas novas tarefas com estes intervalos, cada um tendo como área maior
a área já calculada para estes pela thread. A thread empilha uma das tarefas e retém a outra
como tarefa vigente, para minimizar os acessos à memória. Após isso ela retorna ao começo
do laço de repetição.
\item Se a diferença for pequena o suficiente, a thread soma a área maior ao seu acumulador local
e retorna ao início do laço de repetição.
\end{itemize}
\end{itemize}

Ao término de todas as threads a função envelope soma os resultados parcias ao resultado total, e
retorna este valor.

\subsubsection{A estrutura}
\label{sec:org16993ef}
\begin{verbatim}
// Estrutura de tarefa representando os intervalos
typedef struct tarefa {
  double l;
  double r;
  double area_maior
} tarefa_t;

// Estrutura de pilha com coleção implementada através de vetor dinâmico
typedef struct pilha {
  tarefa_t *colecao;
  int indice;
  int tam;
  int cap;
} pilha_t;

// Função de inicialização da estrutra de pilha e de sua coleção
void p_init(pilha_t **);
// Função chamada ao se tentar inserir numa pilha cheia
// realoca a coleção com o dobro de capacidade
void p_cheia(pilha_t *);
// Retorna 1 se a pilha estiver vazia, 0 caso contrário
int p_vazia(pilha_t *);
// Insere uma tarefa no topo da pilha
p_insere(pilha_t *, tarefa_t);
// Retira uma tarefa do topo pilha
tarefa_t p_retira(pilha_t *);
// Libera a memória da coleção e da estrutura da pilha
void p_destroi(pilha_t *);
\end{verbatim}
\captionof{figure}{\label{pilha.h}
Cabeçalho da estrutura auxiliar de pilha}

\subsection{Divisão inicial de tarefas}
\label{sec:org6ec8f67}
Este algoritmo é significativamente mais simples e, embora não haja como garantir o balanceamento
de carga, se mostrou a solução com melhor desempenho na maioria dos casos.

A função envelope divide o intervalo inicial \([a, b]\) em \(n\) subintervalos, onde \(n\)
é o número de threads. Cada thread receberá uma dessas divisões, e executará o algoritmo
recursivo sequencial com este intervalo como ponto de partida. O resultado final é guardado no
vetor global, e a função envelope irá somar os resultados parciais ao resultado total ao término
de cada thread. Este valor é finalmente retornado pela função.

\section{Conclusão}
\label{sec:org8445e7e}
Ao longo do projeto, percebeu-se uma grande dificuldade em implementar um algoritmo concorrente que
demonstrasse um desempenho superior ao sequencial. As soluções encontradas eram frequentemente
muito complexas ou relativamente ineficientes se comparadas com o algoritmo sequencial.

Apesar disso, o algoritmo utilizando a divisão inicial de tarefas se revelou capaz de apresentar
um grande ganho de desempenho, que se mostrou ainda mais consistente ao se fixar erros máximos
cada vez menores, ou intervalos de cálculo particularmente complexos.

\subsection{Ganho}
\label{sec:orgf5e10a3}
Efetuando-se os testes com intervalos \([-0.9, 0.9]\), \([-10, 10]\) e \([-15, 15]\), com erro
máximo fixado em \(10^{-10}\), os seguintes ganhos foram obtidos em relação ao algoritmo sequencial:

\subsubsection{Intervalo \([-0.9, 0.9]\)}
\label{sec:org50698e2}
\begin{center}
\begin{tabular}{lrrrrrrr}
 & \(f_a\) & \(f_b\) & \(f_c\) & \(f_d\) & \(f_e\) & \(f_f\) & \(f_g\)\\
\hline
Pilhas & 0.0057 & 0.9935 & 1.0147 & 1.2571 & 1.0859 & 1.0996 & 1.7114\\
Divisão de Tarefas & 0.0015 & 2.1805 & 2.4577 & 0.7960 & 1.9024 & 0.4080 & 2.8796\\
\end{tabular}
\end{center}

\subsubsection{Intervalo \([-10, 10]\)}
\label{sec:org479eb49}
\begin{center}
\begin{tabular}{lrrrrrrr}
 & \(f_a\) & \(f_b\) & \(f_c\) & \(f_d\) & \(f_e\) & \(f_f\) & \(f_g\)\\
\hline
Pilhas & 0.0066 & NaN & 0.8304 & 0.7522 & 0.6165 & 0.6215 & 0.9053\\
Divisão de Tarefas & 0.0032 & NaN & 4.5189 & 2.0445 & 0.9615 & 0.9807 & 1.0312\\
\end{tabular}
\end{center}

\subsubsection{Intervalo \([-15, 15]\)}
\label{sec:org7a3dcb6}
\begin{center}
\begin{tabular}{lrrrrrrr}
 & \(f_a\) & \(f_b\) & \(f_c\) & \(f_d\) & \(f_e\) & \(f_f\) & \(f_g\)\\
\hline
Pilhas & 0.0075 & NaN & 0.7992 & 0.7174 & 0.5672 & 0.4924 & 0.6787\\
Divisão de Tarefas & 0.0037 & NaN & 2.4278 & 1.7205 & 1.0164 & 1.0134 & 1.0043\\
\end{tabular}
\end{center}

\subsection{Estratégias não exploradas}
\label{sec:org26468d5}
Dentre as outras estratégias que não nos levaram a conclusões diferentes das obtidas até aqui,
podemos citar o uso de estruturas diferentes da pilha de vetor dinâmico, como pilhas encadeadas
(que possuem um desempenho muito inferior) e pilhas de vetor estático (menos flexíveis e com quase
o mesmo desempenho), e filas.

Além disso também houve uma tentativa de usar uma estratégia de quadratura adaptativa "global", com
a diferença entre as áreas sendo calculada em função da aproximação da área no intervalo original inteiro,
a cada nova divisão dos subintervalos, que também não apresentou um grande ganho no desempenho.

\subsection{Considerações finais}
\label{sec:orga4b23ba}
Considerando os métodos encontrados para a solução do problema, a técnica mais eficiente se mostrou o uso
da divisão inicial de tarefas entre as threads, com o processamento recursivo ocorrendo em seus fluxos de
execução.

Embora não tenha sido possível garantir o balanceamento de carga, deve-se levar em consideração que
a principal motivação para se procurar o balanceamento é maximizar o desempenho, efeito que foi
obtido com a divisão inicial de tarefas.

Além disso existem outras técnicas não exploradas neste trabalho por dificuldades ao se implementá-las,
que podem ou não ter um desempenho ainda superior ao encontrado, fazendo um uso mais eficiente da memória
e de técnicas de concorrência fora da alçada de conhecimento atual dos participantes.

Apesar disso, a solução obtida já é evidente para indicar que o uso de técnicas de programação
concorrente é vantajoso ao ser combinado com métodos numéricos de cálculo, uma vez que estes métodos
com frequência se baseiam em numerosas iterações e outras técnicas que apresentam potencial de
ganho de eficiência com o uso de múltiplos núcleos de processamento.
\end{document}
